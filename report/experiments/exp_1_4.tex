In the fourth experiment, we measure the time of task creation, both for ordinary processes and for kernel threads, and compare them.

To measure ordinary process creation time, we created a recursive program.
The basic idea is that the program relaunches itself using the {\tt excv} function.  
To ensure the program runs exactly 100 times, the program passes an argument {\tt trial} which is assumed to be 0 on first run and is incremented each re run.
Before recursing, the program resets the processor cycle counter to 0 and then the first thing the new process does is check the processor cycle counter.
This cycle count measures how many cycles occured between the {\tt excv} call and the first line of execution for the new process.
Thus we measure the time to create the new process.

To measure kthread creation time, we developed a kernel module and load it using insmod.
Upon loading, the loops 100 times, each time resting the cycle counter, starting a thread, and within the thread printing the value of the cycle counter.
To ensure the thread runs immediately (since we have a single core cpu), the module's main thread is told to sleep briefly after the thread is created.

The results of these two experiments are tabulated in Table \ref{tab:exp1_4}.
As you can see it takes about 3 million cycles to launch a process and 170k cycles to launch a kernel thread.
This is in line with expectations of a process taking significantly longer than a thread.

\begin{table}
\begin{tabular}{|c|c|c|}\hline
& Mean & Standard Deviation \\\hline
Process &  3100000 & 193338 \\
Kthread & 170000 & 72571 \\\hline
\end{tabular}
\caption{Results of process creation experiment}\label{tab:exp1_4}
\end{table}