In the fourth experiment, we wished to measure the time of task creation, both for ordinary processes and for kernel threads, and to compare them. Unfortunately, time did not permit for testing of kernel threads, but we were able to obtain measurements for task creation time for ordinary processes. This was accomplished using the method on the next page (note that the semantics of ``execute" are slightly different than in previous sections of this paper; instead of merely executing the actions under observation, relying on external variables for timing, the execute method took the measurements internally and returned the meaningful change in time to the running program):

\newpage

\noindent \begin{verbatim}
unsigned long execute() {
  
  // initialize pipes for interprocess communication
  int pfd[2];
  pipe(pfd);

  unsigned long t_start, t_end, ctime;

  t_start = ccnt_read(); //get time before fork
  if(fork()) { // parent
    t_end = ccnt_read();  // time of fork return
    close(pfd[1]);  // only receiving messages from child; close output pipe

  else { //child
    t_end = ccnt_read();
 
    // only sending messages to parent; close input pipe
    close(pfd[0]);

    // send time of process start
    write(pfd[1], (char*)&t_end, sizeof(unsigned long));

    // close pipe and exit
    close(pfd[1]);
    exit(0);
  }

  // if child ran first, then update t_end
  if(ctime < t_end) t_end = ctime;
  
  // account for looped clock (needs more robust handling in future)
  if(t_end < t_start) t_end += 1000000000; // clock loops every billion ticks

  return t_end - t_start;
}
\end{verbatim}

\newpage

The idea behind this method is that the child process sends the time of its first instruction $-$ the acquisition of the time $-$ to the parent process. The parent process then calculates the difference between the time taken before the fork call and the time at which the child process began executing, and returns it.

After gathering the data, we found an unreasonably high mean time, due to unreasonably large outliers which are possibly caused by some overlooked cases below. Most times were in the range of 21000-24000ns, but some were as high as 100000000ns. Of 1000000 trials, 824856 were ``inliers", defined (admittedly ad-hoc) by being less than 50000ns. Of these inliers, the mean was much more representative: 26613ns.

In hindsight, the method above is not optimal for a number of reasons. First, our method for dealing with the ambiguity about which process, parent or child, assumes control of the processor after the ``fork" was not adequate. Particularly, if in one instance the parent assumes control after ``fork", we assumed that the child process had been created and was immediately ready to be scheduled. In fact, we could not find justification for this assumption online $-$ it may be the case that the child process may be incompletely created or scheduled. Fortunately, this does not seem to have been a significant factor in the accuracy of our results $-$ of 1000000 trials, the child process assumed control of the processor 924669 times. In the future, however, we will only record data from trials in which this is the case, and get rid of corrupting data from the other trials (in which the parent began executing, and therefore the time returned by the child did not accurately reflect the time of task creation). An easier solution might be to use ``vfork", which guarantees that the parent processor will not gain access to the CPU after spawning the child process.

Second, we should have explored a more holistic set of process-creation primitives, such as ``exec", ``vfork", and ``clone". It may turn out that ``fork" is not a good representative of process creation in general, although it may be the most invoked function for process creation.