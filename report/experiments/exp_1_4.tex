In the fourth experiment, we wished to measure the time of task creation, both for ordinary processes and for kernel threads, and to compare them. Since the last milestone, we have implemented code for testing kernel thread creation time (given below), but unfortunately, we have had difficulty with compiling kernel modules. 
\newline
\newline
Ordinary process-creation time was measured through successive ``execv" calls. This method could not make use of the experimental framework described in the introduction to the experiments section, because the ``execv" call kills the calling process on success. Instead, we maintain state (trial number, time of exec call) through clever use of command line arguments, as shown below:

\noindent \begin{verbatim}
int main(int argc, char *argv[]) {

  GET_HIGH(end);

  if(argc == 1) { // initial run, not a trial
    newargs[0] = process_name;
    newargs[1] = "1"; // trial number
    newargs[2] = [allocate space for time measurement and null terminator]
    newargs[3] = [NULL STRING];

    GET_HIGH(*(unsigned long *)newargs[2]); // lol
    // run first trial
    execv(process_name, newargs);
  }

  // check current trial; if >= MAX_N, exit
  trial = argv[1];
  if(trial == MAX_N) exit(0);

  // get passed time
  start = argv[2];
  fprintf(datafile, "%ul\n", end - start);

  // set up next trial
  newargs[0] = process_name;
  newargs[1] = "[trial+1]";
  newargs[2] = [allocate space for time measurement and null term]
  newargs[3] = [NULL STRING];
  GET_HIGH(*(unsigned long *)newargs[2]);
  execv(process_name, newargs);

}
\end{verbatim}

\newpage

Each time the program is run, its command line arguments inform it of 1) whether it is running a trial and if so 2) which trial it is executing, and finally of 3) when the exec call that initiated it was made, so that it can calculate the time it took for it to be created.
\newline
\newline
This time turns out to be quite large $-$ 3200000, with a standard deviation of 130000. 
\newline
\newline
As for kernel threads, we can only speculate at this point, but we imagine that kernel threads will be faster to create, since enjoy higher priority. The code to test this is as follows (it had to be custom, and could not take advantage of the {\tt basic\_experiment} framework):

\begin{verbatim}
int reply_with_time(void *data) {
  GET_HIGH(recv_time);
  write(c2p[1],"0", 1);
  return 0;
}

int exp_init() {
  for(i = 0; i < MAX_N; ++i) {
    reset();
    kthread_create(&reply_with_time,0,"0");
    read(c2p[0], &c, 1);
    fprintf(datafile, "%ul\n", recv_time);
  }

  return 0;
}
\end{verbatim}

\noindent Here, the parent thread creates a child thread; reads from a pipe that the child writes to, in order to block as soon as possible; the child then records the time it began executing, and exits; and the parent resumes and records the time of receipt in the datafile. 
\newline
\newline
While this method is less than ideal $-$ the parent may resume execution for a short time before blocking and allowing the child thread, which may have been created but not queued, to record its start time $-$ we could not find a way around this. It is possible that we could just take the time between the start and the end of the {\tt kthread\_create} call, but if the child started first, then this would become inaccurate. Perhaps taking the minimum of both measurements will yield a more accurate measurement of kernel thread creation time.