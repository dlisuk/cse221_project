The third experiment we ran was to measure the amount of time taken by making an operating system call.
To do this we wish to execute a call which will do very little without being cached.  
We decided to use the {\tt getcpu} system call.

We implement this experiment as follows: 
\begin{verbatim}
unsigned long measure(){
  int rc;
  GET_HIGH(start);
  rc = syscall(SYS_getcpu);
  GET_HIGH(end);
  return end - start;
}
\end{verbatim}

We expect this to be about two orders of magnitude larger than the basic overhead measure since we have to switch to the OS, do some work, and then switch back to our user program.  We take the sum of the procedure call overhead for a single argument and the overhead for returning from a function, and multiply by 100, to get $(19+16) \times 100 = 2500$ cycles.

The results of the experiment are"

\begin{table}[h]
\centering
\begin{tabular}{|c|c|}\hline
Mean & 2111 cycles \\
SD & 165 cycles\\\hline
\end{tabular}
\end{table}

This is roughly what we predicted, though we were off by about 3 standard deviations.