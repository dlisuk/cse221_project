The third experiment we ran was to measure the amount of time taken by making an operating system call.
To do this we wish to execute a call which will do very little without being cached.  
We decided to use the {\tt getcpu} system call.

The {\tt getcpu} system call accepts three pointers and writes data into the memory locations pointed to and then returns a single int.
Thus it dose very little work within the kernel and will be dominated by the entering/exiting of kernel mode.
We are unsure exactly what the time of a system call will be but expect it to be approximately two orders of magnitude higher than a procedure call (since it requires popping the current state to the stack, switching stacks, etc).
Thus we suspect the time will be between 500 and 3000 cycles.

We find the procedure call to take an average of 2100 cycles with a standard deviation of 165 cycles.
This is in line with our rough estimations based on magnitude estimations.
Since the time doesn't significantly decrease after a few calls, the call does no caching (We tried some other sys calls and saw massive drops in delay after the first couple calls, indicating they were being cached.
Since there is no NULL syscall that always calls into the kernel and does no work,  we are unable to be certain that this is the base overhead for a system call; however, it seems like one of the simplest ones without caching.
An improved experiment would implement a kernel module that adds a NULL call for this experiment; however, this looks to be a substantial effort.