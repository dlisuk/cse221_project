The first experiment measures the number of cycles passed between resetting the instruction counter and then reading it.
This time is the overhead associated with measuring and will be overhead for all experiments.

This experiment executes the {\tt RESET} macro and then the {\tt GET\_HIGH} macro on the next experiment.
The {\tt RESET} sets the clocks to 0 and then runs the serializing instruction to prevent OOO execution.
Since the serializing instruction and clocking will be pipelined one cycle will pass after the zeroing.
The {\tt GET\_HIGH} executes the serializing instruction (4 cycles from issue to execution), loads the destination variable from memory to a register(3 cycles according to documentation), then copies from the counter register to the variable register (4 cycles from start to register read due to pipeline).
After the register read the remaining work (transferring variable from register back into memory) will be outside of the measured time.
Thus we expect 12 cycles to pass from zeroing the counter till reading the counter.

Our experimental results found a mean time of 11 cycles with a standard deviation of 0.
This is one cycle shorter than our prediction.
It is probable that the destination reading and the counter copying will be pipelined so the copy takes is fetched while the memory is being read. 
