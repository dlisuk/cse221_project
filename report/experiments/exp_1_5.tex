
In this experiment, we measure the time taken for a context switch to occur, both for ordinary processes and kernel threads. 

To measure process context switch, our program forks a child process and creates two blocking pipes between the two.
The parent writes and the child reads from pipe 1 and the opposite for pipe 2.
To measure performance, the parent sends a message to child, forcing a context switch.
The child then resets the instruction counter and sends a message to the parent, forcing a second context switch.
The parent prints the instruction counter.
This process cycles 100 times to get sufficient measurements.

To measure kernel context switching we developed a kernel module which spawns a single separate kthread and then force 100 context switches back and forth and measure the switching time.
Since kernel modules cannot use libc and thus piping, we had to get creative in forcing context switches. 
We utilize a single mutex lock which is locked by the child and unlocked by the parent.
Initially the lock is locked and the child thread is created and waits since it tries to lock the locked lock.
The parent then resets the process counter, unlocks the lock, and uses {\tt wake\_up\_process} on the child and goes to sleep.
The child should then gets the lock and prints the process counter and then tries to re lock.
This process cycles 100 times to get sufficient measurements.

The results of these two experiments are tabulated in Table \ref{tab:exp1_5}.
The process context switch time is about 10x slower than the kernel thread context switch time.
This is inline with our expectations.

\begin{table}
\begin{tabular}{|c|c|c|}\hline
& Mean & Standard Deviation \\\hline
Process &  25000 & 5018\\
Kthread & 5200 & 561\\\hline
\end{tabular}
\caption{Results of context switch experiment}\label{tab:exp1_5}
\end{table}