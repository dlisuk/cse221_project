To run our experiments we used a Raspberry Pi model B single board computer.   
This is a simple ARM 6 based computer which is availible for under \$50.
We run a Pi-targeted Debian derivative known as Raspbian on it for all performance measurements.
See table \ref{tbl:specs} for details on the hardware specs of the machine.

\begin{table}[h]
\centering
\begin{tabular}{|l|l|}
\hline
\multicolumn{2}{|c|}{\textbf{CPU}}              \\ \hline
Type                 & ARM 6 Compatible         \\ \hline
Model                & ARM1176JZF-S             \\ \hline
Cycle Time           & 1.4ns                    \\ \hline
L1 Cache             & 16KB Data, 16KB Instructions          \\ \hline
L2 Cache             & 128KB           \\ \hline
\multicolumn{2}{|c|}{\textbf{Memory}}           \\ \hline
Size                 & 512MB(CPU+GPU)           \\ \hline
Bus                  & DDR2                     \\ \hline
\multicolumn{2}{|c|}{\textbf{IO Bus}}           \\ \hline
Speed                & Unknown                  \\ \hline
\multicolumn{2}{|c|}{\textbf{Disk}}             \\ \hline
Type                 & Class 4 SD Card          \\ \hline
Size                 & 16GB                     \\ \hline
\multicolumn{2}{|c|}{\textbf{Network Card}}     \\ \hline
Speed                & 100MBPS                  \\ \hline
\multicolumn{2}{|c|}{\textbf{Operating System}} \\ \hline
Type                 & Linux                    \\ \hline
Distribution       & Raspbian(Debian)\\ \hline
Version             & Wheezy\\ \hline
Kernel Version  & 3.12.30+ \\\hline
\end{tabular}
\caption{Table of specs for the Rasberry Pi Model B\cite{infosheet}\cite{rpicomp}\cite{elinux}}
\label{tbl:specs}

\end{table}

For the network tests and the remote file system tests, we had to connect to a remote server.
To do this we employed a early 2014 model Macbook Pro 13 inch with specs described in Table \ref{tbl:macbookspecs}.
The Pi connected to the lan via a wired gigabit ethernet port and the Mac through 802.11G (54 MB/s).
Only one hop existed between the two machines.

\begin{table}[h]
\centering
\begin{tabular}{|l|l|}
\hline
\multicolumn{2}{|c|}{\textbf{CPU}}              \\ \hline
Type                 & Intel Core i5         \\ \hline
Processor Speed & 2.4Ghz                    \\ \hline
L1 Cache             & 32KB Data, 32KB Instructions          \\ \hline
L2 Cache             & 256KB           \\ \hline
L3 Cache             & 3MB           \\ \hline
\multicolumn{2}{|c|}{\textbf{Memory}}           \\ \hline
Size                 &  8GB          \\ \hline
Bus                  & DDR3                     \\ \hline
\multicolumn{2}{|c|}{\textbf{Disk}}             \\ \hline
Type                 & SSD           \\ \hline
Size                 & 251GB                     \\ \hline
\multicolumn{2}{|c|}{\textbf{Network Card}}     \\ \hline
Type                 & Wireless 802.11ac \\ \hline
Speed                &1 Gigabit                  \\ \hline
\multicolumn{2}{|c|}{\textbf{Operating System}} \\ \hline
Type                 & Mac OSX                      \\ \hline
Version             & 10.9.5 \hline
\end{tabular}
\caption{Table of specs for Macbook pro used for the server in network and remote file system tests.  Data acquired from system information app.}
\label{tbl:macbookspecs}
\end{table}