The file system has several layers of caching which presents a problem for any attempt to measure performance of the underlying file system.
The lowest cache is the disk's data cache, then the operating system provides a layer of cacheing, and finally typical glibc file operations (fread, fopen, etc) provides a third layer of caching.
Since our goal is to measure the raw system performance as much as possible we use the raw OS interface (open, read, etc) instead of the more standard glibc operations.
Additionally, we utilize the {\tt O\_DIRECT} flag to bypass the OS caching system as much as possible.
Additionally we use the {\tt O\_SYNC } flag to ensure syncronous reading to ensure we are  measuring complete reading time.

All reads are performed a single file system block of 4096 bytes at a time.
This is the smallest unit of storage readable using the direct disc access system and thus the most sensible thing to measure.

One thing of note is that our disk is not a typical hard drive but instead a class 10 SD card.
Random reading benchmarks \cite{sdcard_2} have shown random read times for a 4K file system block to be on the order of 4-5 MB/s and sequential read times to be about 20 MB/s.