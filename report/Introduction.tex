The Rasberry Pi is a single board computer availible for under \$50 which is capable of running Linux distributions made for the ARM architecture.  
Originally targeted as an embedded computing learning aid, it has taken the world by storm with where an enthusiast community sprung up and utilized it for media centers, simple servers, robotics control boards, etc.
In this project, we obtain a performance profile of the Rasberry Pi model B computer, the flag ship ``full featured'' model.

Our team consists of Joshua Marxen and David Lisuk. 
David primarily did disk and memory experiments while Josh primarily did OS Services and Network.
However, both members contributed to all sections and the work distribution was approximately even.

Our compiler is gcc version 4.8.3, as implemented in the Raspbian linux operating system.
We disabled any compiler optimizations by compiling with the -O0 flag.
By analyzing the generated assembly code, we have found this optimization level to generate code close to the expected behavior and thus anticipate predictions to line up with measurements.

We evaluate the performance of 4 distinct hardware groups: CPU and OS Services, Memory, Network, and Disk.
For the rest of this paper, we first describe the Pi's hardware and software stack.
Then we describe our experiments by describing the general methodology then each of our four hardware groups separately.